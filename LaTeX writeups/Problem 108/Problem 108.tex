\documentclass[11pt, oneside]{article}   	% use "amsart" instead of "article" for AMSLaTeX format
\usepackage[margin=0.75in]{geometry}                		% See geometry.pdf to learn the layout options. There are lots.
\geometry{letterpaper}                   		% ... or a4paper or a5paper or ... 
%\geometry{landscape}                		% Activate for rotated page geometry
%\usepackage[parfill]{parskip}    		% Activate to begin paragraphs with an empty line rather than an indent
\usepackage{graphicx}				% Use pdf, png, jpg, or eps§ with pdflatex; use eps in DVI mode
								% TeX will automatically convert eps --> pdf in pdflatex		
\usepackage{amssymb}
\usepackage{amsmath}
\usepackage{mathtools}

\usepackage{xcolor}
\usepackage{framed}


\definecolor{shadecolor}{RGB}{0,113,186}
\usepackage{xparse}
\NewDocumentCommand{\DIV}{om}{%
  \IfValueT{#1}{\setcounter{#2}{\numexpr#1-1\relax}}%
  \csname #2\endcsname
}


\title{Project Euler, Problem 108/110:\\\large{Diophantine Reciprocals}}
\author{John Butler}
\date{}							% Activate to display a given date or no date


\begin{document}
\maketitle

\begin{enumerate}
  \item Analogous Equation\\
  In this section, I will rearrange the equation to give us something of the form $y=f(x,n)$.
  \begin{align*}
    \frac1x+\frac1y&=\frac1n\\
    \frac{x+y}{xy}&=\frac 1n\\
    (x+y)n&=xy\\
    xn+yn-xy&=0\\
    y(n-x)&=-xn\\
    y&=\frac{nx}{x-n}
  \end{align*}
  
  \item Proof that the lines $x=n$ and $y=n$ are assymtotes of our new equation with $n$ held constant.\\

    Let $x=n + \delta$. This means that $x = n$ is equivilant to $\delta = 0$, and so we get 
    \begin{align*}
      y &= \frac{n(n+\delta)}{(n + \delta) - n}\\
      &=\frac{n^2+\delta n}{\delta}\\
      &=\frac{n^2}\delta + \frac{\delta n} \delta\\
      &=\frac{n^2}\delta + n)\\
      &=\frac{n^2}\delta + n\\
    \end{align*}
    This is clearly asymtotic at $\delta=0$ and continuous everwhere else.

    Now, clearly the original equation was symmetric, so our new, equivilant equation must be aswell, so the same reasoning must hold for $y=n$.

  \item Symmetric Points\\
    I claim that for every solution $(x,y)$ is a family of 4 ``symmetric points'' for any fixed $n$.
    These points are $\left\{(x,y), (y,x), (2n-x,2n-y),(2n-y,2n-x) \right\}$.
    Notice that when $x=y$, then $(x,y)=(y,x)$ and $(2n-x,2n-y)=(2n-y,2n-x)$, so there are technically only 2 distinct symmetric points in that case.

    Assume that $(x,y)$ is a solution; that is $\frac1x + \frac 1y=\frac 1n$. But is $(2n-x,2n-y)$ a solution? Well if it is, then
    \begin{align*}
      \frac1{2n-x}+\frac1{2n-y}&=\frac 1n\\
      \frac{2n-y}{(2n-x)(2n-y)} + \frac{2n-x}{(2n-x)(2n-y)}&=\frac 1n\\
      \frac{2n-y + 2n-x}{(2n-x)(2n-y)}&=\frac 1n\\
      \frac{4n-(x + y)}{(2n-x)(2n-y)}&=\frac 1n\\
      n(4n-(x+y))&=(2n-x)(2n-y)\\
      4n^2-n(x+y)&=4n^2-2nx-2ny + xy\\
      4n^2-n(x+y)&=4n^2-2n(x-y) + xy\\
      n(x+y)&=xy\\
      \therefore \frac{n(x+y)}{nxy}&=\frac{xy}{nxy}\\
      \frac{x+y}{xy}&=\frac{1}{n}\\
      \frac1y+1x&=\frac1n \text{ which is true by assumption.}
    \end{align*}
    So $(2n-x,2n-y)$ is another solution.

    For the other two points, as previously established, the equation is symmetric along $y=x$, so since $(x,y)$ and $(2n-x,2n-y)$ are solutions, so are $(y,x)$ and $(2n-y,2n-x)$,
    giving us the four symmetric points that claimed. I do not have a formal proof that these are the only symmetric points, but by looking at the graph, it's pretty self evident.

  \item search space restricts to just one "region"\\
  Now this means that only finding solutions on one fourth of the curve will generate all valid results.  

  \item Solutions $(x,y)$ where $x,y,n > 0$, then we will always have $x,y > n$.\\
    Assume once again that $(x,y)$ is a valid solution for some fixed $n$.
    
    Assume for a moment that $x < n$ Then $\frac 1n = \frac 1x + \frac 1y > \frac 1n + \frac 1y$. But $y> 0$ so $\frac 1y > 0$, so $\frac 1n < \frac 1n+\frac 1y$.
    This is a contradiction so $x > n$. A symmetric argument shows $y > n$.

    \begin{align*}
      \frac 1y + \frac 1x &= \frac 1n\\
      \therefore \frac d{dx}\left(\frac 1y + \frac 1x\right) &= \frac d{dx}\frac 1n\\
      \therefore -\frac 1{y^2}\frac{dy}{dx} - \frac 1{x^2}&=0\\
      \therefore \frac{dy}{dx}&=-\frac{y^2}{x^2}\\
      y^2,x^2 >* 0\\
      \therefore -\frac{y^2}{x^2} < 0
    \end{align*}
    

  \item If $(x,y)$ is a solution then is $(y,x)$ is also a positive solution, but not $(2n-x, 2n-y)$ and $(2n-y,2n-x)$\\
    $(x,y)$ is a positive solution, so $x, y, n > 0$. It should be quite clear that $(y,x)$ fulfills the same solution.
    Because $(x,y)$ is a positive solution, $x > n \equiv -x < -n \equiv 2n-x < 2n - n \equiv 2n-x < n$. Therefore $(2n-x,2n-y)$ and $(2n-y,2n-x)$ are not positive
    solutions.

  \item Finding all $(x, y)$ where $x \le y$ is enough to generate all unique solutions.\\
    Let's assume that we've generated every solution $(x,y)$ such that $x \le y$. Let's say that there exists a $(x',y')$ is a solution where $x' > y'$. 
    Then $(y',x')$ is a solution, but $y'< x'$, so we generated it by assumption, and $(x',y')$ is not unique.


  \item actual algorithm!\\
    Now I belive we have all of the major peices needed to compose our algorithm.
    There will be an outer loop iterating over each $n$. 
    Here we will reset the counter to 1, because $2n$ is always an integer, and is a special case where $(x,y)$ and $(y,x)$ are not distict solutions.
    There will then be a secondary loop which iterates from $n+1$ to $2n-1$ if $xn$ divides $x-n$, then we've found a solution, 
    and we can add one to the counter.


  \item Performance Analysis\\
    This algorithm has three major parts. The outer loop, the inner loop, and the check done each iteration. The check is done in $O(1)$ time (I think?),
    The inner loop iterates $O(N)$ times where $N$ is the answer we're looking for, and the outer loop runs $O(N)$ times. Therefore the total runtime complexity is $O(N^2)$

\end{enumerate}

\end{document}  
