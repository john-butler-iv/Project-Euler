\documentclass[11pt, oneside]{article}   	% use "amsart" instead of "article" for AMSLaTeX format
\usepackage[margin=0.75in]{geometry}                		% See geometry.pdf to learn the layout options. There are lots.
\geometry{letterpaper}                   		% ... or a4paper or a5paper or ... 
%\geometry{landscape}                		% Activate for rotated page geometry
%\usepackage[parfill]{parskip}    		% Activate to begin paragraphs with an empty line rather than an indent
\usepackage{graphicx}				% Use pdf, png, jpg, or eps§ with pdflatex; use eps in DVI mode
							% TeX will automatically convert eps --> pdf in pdflatex		
\usepackage{amssymb}
\usepackage{amsmath}
\usepackage{mathtools}

\usepackage{xcolor}
\usepackage{framed}


\definecolor{shadecolor}{RGB}{0,113,186}
\usepackage{xparse}
\NewDocumentCommand{\DIV}{om}{%
\IfValueT{#1}{\setcounter{#2}{\numexpr#1-1\relax}}%
\csname #2\endcsname
}


\title{General Proofs}
\author{John Butler}
\date{}							% Activate to display a given date or no date


\begin{document}
\maketitle

\tableofcontents
\newpage

\section{Even and Odd Proofs}
	\subsection{Lemma: $\forall n \in \mathbb{N}$, $\exists k \in \mathbb{N}: n = 2k$ or $n =2k + 1$}
		Proof by induction over $n$:
		\begin{center}
		\begin{tabular}{c|c}
			\underline{case $n = 0$:}&\underline{case $n = 1$:}\\
			$0 = 2\cdot 0$&$1 = 0 + 1$\\
			&$1 = 2\cdot 0 + 1$\\
			Let $k  =0$& Let $k = 0$\\
			$0 = 2k$&$1 = 2k + 1$
		\end{tabular}
		\end{center}
			Assume $n - 1 = 2k$, or $n - 1 = 2k + 1$\\
		\begin{center}
		\begin{tabular}{c|c}
			\underline{case $n - 1 = 2k$:}&\underline{case $n - 1 = 2k + 1$:}\\
			$n = 2k + 1$&$n = 2k + 2$\\
			&$n = 2(k + 1)$\\
			&Let $k' = k + 1$\\
			&$n = 2k'$
		\end{tabular}
		\end{center}

	\subsection{Theorem: $o$ is odd $\iff \exists n \in \mathbb{N} : o =2n + 1$, and $e$ is even $\iff \exists n \in \mathbb{N} : e = 2n$}
		\begin{align*}
			\text{Suppose } e & \text{ is even.}\\
			\iff e & \text{ is divisible by } 2\\
			\iff \frac{e}{2} &= n \text{ for some } n \in \mathbb{N}\\
			\iff e &= 2n
		\end{align*}
		\begin{align*}
			\text{Suppose } o & \text{ is odd.}\\
			\iff o & \text{ is not divisible by } 2\\
			\iff o& \text{ is not even.}\\
			\iff o &\ne 2n\\
			\iff o &= 2n + 1 \text{ by lemma 1.1}
		\end{align*}

	\subsection{Theorem: An odd number plus an even number equals an odd number, and an odd number plus an odd number equals an even number}
		\begin{align*}
			\text{Consider an odd number}&\text{ plus an even number.}\\
			\iff & (2n_1+1) + (2n_2) \text{ for some } n_1, n_2 \in \mathbb{N}\\
			= &\ 2n_1 + 2n_2 + 1\\
			= &\ 2(n_1 + n_2) + 1\\
			\text{Let } n_3 =& n_1 + n_2\\
			= &\ 2n_3 + 1\\
			\text{which }& \text{is odd by lemma 2}.\\
			\therefore \text{ an odd plus an }&\text{even equals an odd.}
		\end{align*}
		\begin{align*}
			\text{Now consider an odd number}&\text{ plus an odd number.}\\
			\iff & (2n_1+1) + (2n_2+1) \text{ for some } n_1, n_2 \in \mathbb{N}\\
			= &\ 2n_1 + 2n_2 + 1 + 1\\
			= &\ 2n_1 + 2n_2 + 2\\
			= &\ 2(n_1 + n_2 + 1)\\
			\text{Let } n_3 = & n_1 + n_2 + 1\\
			= &\ 2n_3\\
			\text{which }& \text{is even by theorem 1.2}.\\
			\therefore \text{ an odd plus an }&\text{odd equals an even.}
		\end{align*}

\section{}
\subsection{$\sum\limits_{n=1}^{N}n= \frac{N(N+1)}{2}$}
	Proof by induction:\\
	Case $N=1$:
	\begin{align*}
		\sum\limits_{n=1}^{N}n &= \sum\limits_{n=1}^{1}n \\
		&= 1\\
		&= \frac 2 2\\
		&= \frac{1\cdot(1 + 1)}{2}\\
		&= \frac{N(N+1)}{2}
	\end{align*}
	Assume $\sum\limits_{n=1}^{N}n= \frac{N(N+1)}{2}$:
	\begin{align*}
		\sum_{n=1}^N n &= \frac{N(N+1)} 2\\
		\left(\sum_{n = 1}^N n\right) + (N + 1) &= \frac {N(N+1)} 2 + (N+1)\\
		\sum_{n = 1}^{N+1} n &= \frac {N ( N+1)} 2 + \frac{2(N + 1)} 2\\
		\sum_{n = 1}^{N+1} n &= \frac {N ( N+1) + 2(N + 1)} 2\\
		\sum_{n = 1}^{N+1} n &= \frac {(N+1)\cdot (N + 2)} 2\\
		\sum_{n = 1}^{N+1} n &= \frac{(N+1)\cdot ((N+1) + 1)} 2
	\end{align*}

	Therefore by induction, $\forall N \in \mathbb{N}, \sum\limits_{n=1}^{N}n= \frac{N(N+1)}{2}$


\end{document}
