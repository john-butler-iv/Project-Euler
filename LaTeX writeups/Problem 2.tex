\documentclass[11pt, oneside]{article}   	% use "amsart" instead of "article" for AMSLaTeX format
\usepackage[margin=0.75in]{geometry}                		% See geometry.pdf to learn the layout options. There are lots.
\geometry{letterpaper}                   		% ... or a4paper or a5paper or ... 
%\geometry{landscape}                		% Activate for rotated page geometry
%\usepackage[parfill]{parskip}    		% Activate to begin paragraphs with an empty line rather than an indent
\usepackage{graphicx}				% Use pdf, png, jpg, or eps§ with pdflatex; use eps in DVI mode
								% TeX will automatically convert eps --> pdf in pdflatex		
\usepackage{amssymb}
\usepackage{amsmath}
\usepackage{mathtools}

\usepackage{xcolor}
\usepackage{framed}


\definecolor{shadecolor}{RGB}{0,113,186}
\usepackage{xparse}
\NewDocumentCommand{\DIV}{om}{%
  \IfValueT{#1}{\setcounter{#2}{\numexpr#1-1\relax}}%
  \csname #2\endcsname
}


\title{Project Euler, Problem 2:\\\large{Even Fibonacci Numbers}}
\author{John Butler}
\date{}							% Activate to display a given date or no date


\begin{document}
\maketitle

\tableofcontents
\newpage

\section{Problem Description}
	Find the sum of all $F_n < 4,000,000$ such that $F_n$ is even, where
	\[ F_n  = \begin{cases}
		0 & n = 0 \\
		1 & n = 1 \\
		F_{n-1} + F_{n-2} & n \ge 2
	\end{cases}	\]

\section{Theorems whose results aid in my solution}
	\subsection{Only every third Fibonacci number is even}
		Proof by induction:\\
		case $n \in \{0, 1, 2\}$
		\begin{center}
		\begin{tabular}{c|c|c}
			$F_0 = 0$&$F_1 = 1$&$F_2 = F_1 + F_0$\\
			&$F_1 = 0 + 1$&\\
			$F_0 = 2\cdot 0$&$F_1 = 2\cdot0 + 1$&$F_2 =  F_1 + 0$\\
			Let $k_1 = 0$&Let $k_2 = 0$&\\
			$\therefore F_0 = 2k_1$&$\therefore F_1 = 2k_2 + 1$&$F_2 = F_1$\\
			$\therefore F_0$ is even&$\therefore F_1$ is odd&$\therefore F_2$ is odd\\
		\end{tabular}
		\end{center}
		Assume $F_{n - 3}$ is even, $F_{n - 2}$ is odd, and $F_{n - 1}$ is odd
		\begin{center}
		\begin{tabular}{c|c|c c}
			$F_n = F_{n - 1} + F_{n - 2}$&$F_{n+1} = F_n + F_{n - 1}$&$F_{n+2} = F_{n + 1} + F_n$\\
			$F_n =$ odd $+ $ odd&$F_{n+1} =$ even $+$ odd&$F_{n+ 2} =$ odd $+$ even\\
			$F_n =$ even&$F_{n+1} =$ odd& $F_{n+2} =$ odd
		\end{tabular}
		\end{center}
		Therefore the first of each three subsequent Fibonacci numbers are even.

	\subsection{$F_n = \frac{(1 + \sqrt 5)^n - (1 - \sqrt 5)^n}{2^n\sqrt 5}$}
		\begin{align*}
			\text{Assume }F_n &= cr^n.\\
			F_n  &= F_{n-1} + F_{n - 2}\\
			\implies cr^n &= cr^{n-1} + cr^{n-2}\\
			  &= cr^n(r^{-1} + r^{-2})\\
			\therefore 1 &= \frac{1}{r} + \frac{1}{r^2}\\
			\therefore r^2& = r + 1\text{ Assuming } r\ne 0\\
			\therefore r^2 - r - 1& = 0\\
			\therefore r &= \frac{1 \pm \sqrt{(-1)^2 - 4\cdot 1 \cdot (-1)}}{2} \text{ using the quadratic formula}\\
			  &= \frac{1\pm \sqrt{5}}{2}\\
			\text{Say }r_1 &= \frac{1+\sqrt 5}{2}\\
			\text{and } r_2 &= \frac{1 - \sqrt 5}{2}
		\end{align*}
		\begin{align*}
			\text{Now assume }F_n &= \alpha r_1^n + \beta r_2^n\\
			F_0 &= 0\\
			\therefore 0& = \alpha r_1^0 + \beta r_2^0\\
			& = \alpha + \beta\\
			\therefore \alpha &= -\beta\\
			F_1& = 1\\
			\therefore 1 &= \alpha r_1^1+\beta r_2^1\\
			  &= \alpha r_1+\beta r_2\\
			 & = -\beta r_1 + \beta r_2\\
			 & = \beta(r_2 - r_1)\\
			 & = \beta\left(\frac{1 - \sqrt 5}{2} - \frac{1 + \sqrt 5}{2}\right)\\
			  &= \beta\left(\frac{1 - \sqrt 5 -1 - \sqrt 5}{2}\right)\\
			 & = \beta\left(-\frac{2\sqrt 5}{2}\right)\\
			 1& = \beta\left(-\sqrt 5\right)\\
			\therefore \beta &= -\frac{1}{\sqrt 5}\\
			\therefore \alpha &= \frac{1}{\sqrt5}\\
			\therefore F_n &= \frac{1}{\sqrt 5}\cdot \left(\frac{1 + \sqrt 5}{2}\right)^n - \frac{1}{\sqrt 5} \cdot\left( \frac{1 - \sqrt 5}{2}\right)^n\\
			&= \frac{(1 + \sqrt 5)^n - (1 - \sqrt 5)^n}{2^n\sqrt 5}
		\end{align*}

\section{Application to Code}
	Now that we know that every third Fibonacci number is even, we don't have to bother checking if a number is even. Instead, since we know that $F_0 = 0$ is the first Fibonacci number which is even, we can add $F_0$ to our sum, then compute $F_1$, $F_2$, and then add $F_3=2$ to our sum, and so on. Granted, checking if a number is even is computationally light.\\

	Additionally, $F_0$ doesn't affect our sum, so we can start with $F_3=2$; Project Euler begins the Fibonacci sequence at $F_2$ in the problem description, reinforcing this fact.  


\section{Comlexity Analysis/ Further Optimizations}
	For this section, assume that $n$ is the index of the maximum Fibonacci number we must execute, and $L$ is the limit ($4,000,000$)\\

	Considering the definition, it calculating the Fibonacci numbers would seem to be most natural to use a top-down recurrsive algorithm, but this would be incredibly slow, $O(2^n)$ for \textit{each} number, meaning we end up with a time complexity of $O(n2^n)$. Of course, this is a common example used by programming courses for recursion, so this may be many peoples' first approach. There are certainly optimizations to this algorithm, such as storing the result of previous computations, so we don't have to recompute $F_{n-1}$ and $F_{n-2}$, bringing time complexity down to $O(1)$, but there is still a linear space complexity. We don't need to store $F_{n-3}$, since we're always only computing the next number. Fully optimized this way, the time complexity should be $O(n)$ and the space complexity should be $O(1)$\\

	The algorithm I'm using uses a bottom up approach, which also has a space complexity of $O(1)$ and a time complexity of $O(n)$. The idea is to store only $F_{n-1}$ and $F_{n-2}$, so that we can compute $F_n$. After that we shift over what we're storing, so that we forget $F_{n-2}$ and keep $F_n$ and $F_{n-1}$ in order to calculate $F_{n+1}$.
\end{document}  
